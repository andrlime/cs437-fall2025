% packages
\documentclass[b5paper]{article}
\usepackage[utf8]{inputenc}
\usepackage[margin=1.1in]{geometry} % 0.9 for pset, 1.4 for livetex notes

% Math libraries
\usepackage{amsmath}
\usepackage{amssymb}
\usepackage{amsthm}
\usepackage{amsthm}

% Formatting
\usepackage{titling}
\usepackage{fancyhdr}
\usepackage{graphicx}
\usepackage{float}
\usepackage{titlesec}
\usepackage{setspace}
\usepackage[titles]{tocloft}

% Additional commands
\usepackage{cancel}
\usepackage{enumitem}
\usepackage{xfrac}
\usepackage{mathtools}

% Graphcs
\usepackage{tikz}
\usepackage{tcolorbox}
\usepackage{xcolor}
\usepackage{booktabs}
\usepackage{longtable}

% Content
\usepackage{physics}
\usepackage[version=4]{mhchem}
\usepackage{siunitx}
\usepackage{sectsty}
\usepackage{hyperref}
\usepackage[vlined,ruled,linesnumbered,rightnl]{algorithm2e}

\hypersetup{
    colorlinks=true,
    linktoc=all,
    linkcolor=black,
}

\allsectionsfont{\sffamily}
\renewcommand{\cftsecfont}{\normalfont\bfseries\sffamily}
\renewcommand*{\contentsname}{Table of Contents}

\newcommand*\circled[1]{\tikz[baseline=(char.base)]{
            \node[shape=circle,draw,inner sep=2pt] (char) {#1};}}
\renewcommand\real{\mathbb{R}}
\newcommand\complex{\mathbb{C}}
\newcommand\integer{\mathbb{Z}}
\newcommand\rational{\mathbb{Q}}
\renewcommand\natural{\mathbb{N}}
\newcommand\probability{\mathbb{P}}
\newcommand\field{\mathbb{F}}
\newcommand\qsp{\qq{}}
\newcommand\curlyf{\mathcal{F}}
\newcommand\psp{(\Omega, \curlyf, \probability)}
\DeclarePairedDelimiter{\ceil}{\lceil}{\rceil}

\renewcommand{\equation}[1]{
  \begin{align}
    #1
  \end{align}
}
\newcommand{\e}[1]{
  \begin{align}
    #1
  \end{align}
}

\definecolor{DARKBLUE}{HTML}{005E7A}
\definecolor{BLACK}{HTML}{121212}
\definecolor{DARKRED}{HTML}{83223C}
\definecolor{ORANGE}{HTML}{F0B12C}
\definecolor{CYAN}{HTML}{2CF0CB}
\definecolor{RED}{HTML}{C41E3A}

\newcounter{lemmacounter}
\newcounter{definitioncounter}
\newtcolorbox{squarebox}{
  sharp corners,
  colback=yellow!5!white,
  colframe=DARKBLUE!75!BLACK,
  boxrule=0.5px
}

\newtcolorbox{squarebox2}{
  sharp corners,
  colback=yellow!5!white,
  colframe=ORANGE!75!BLACK,
  boxrule=0.5px
}

\newtcolorbox{roundbox1}{
  colback=yellow!5!white,
  colframe=BLACK!75!BLACK,
  boxrule=0.5px
}

\newtcolorbox{roundbox2}{
  colback=yellow!5!white,
  colframe=DARKRED!75!BLACK,
  boxrule=0.5px
}

\newtcolorbox{roundbox3}{
  colback=yellow!5!white,
  colframe=CYAN!75!BLACK,
  boxrule=0.5px
}

\newtcolorbox{roundbox4}{
  sharp corners,
  colback=RED!5!white,
  colframe=RED!75!BLACK,
  boxrule=0.5px
}

\newenvironment{problem}
  {\begin{roundbox1}}
  {\end{roundbox1}}
  
\newenvironment{solution}
  {\begin{proof}[Solution]}
  {\end{proof}}

\newenvironment{definition}
  {\begin{roundbox1}
  \stepcounter{lemmacounter}
  \textcolor{BLACK}{\textsf{\textbf{Definition \thesection.\arabic{lemmacounter}}}}}
  {\end{roundbox1}}

\newenvironment{example}
  {\begin{roundbox2}
  \stepcounter{lemmacounter}
  \textcolor{DARKRED}{\textsf{\textbf{Example \thesection.\arabic{lemmacounter}}}}}
  {\end{roundbox2}}

\newenvironment{lemma}
  {\begin{squarebox2}
  \stepcounter{lemmacounter}
  \textcolor{ORANGE}{\textsf{\textbf{Lemma \thesection.\arabic{lemmacounter}}}}}
  {\end{squarebox2}}

\newenvironment{aside}
  {\begin{roundbox4}
  \textcolor{RED}{\textsf{\textbf{Note}}}}
  {\end{roundbox4}}
  
\newenvironment{theorem}
  {\begin{squarebox}
  \stepcounter{lemmacounter}
  \textcolor{DARKBLUE}{\textsf{\textbf{Theorem \thesection.\arabic{lemmacounter}}}}}
  {\end{squarebox}}
  
\newenvironment{proposition}
  {\begin{roundbox3}
  \stepcounter{lemmacounter}
  \textcolor{DARKBLUE}{\textsf{\textbf{Proposition \thesection.\arabic{lemmacounter}}}}}
  {\end{roundbox3}}

\makeatletter
\DeclareRobustCommand{\@seccntformat}[1]{%
  \def\temp@@a{#1}%
  \def\temp@@b{subsubsection}%
  \ifx\temp@@a\temp@@b
  \textcolor{DARKRED}{\csname the#1\endcsname}%
  \quad
  \else
  \textcolor{DARKBLUE}{§\csname the#1\endcsname}%
  \quad
  \fi
} 
\makeatother

\numberwithin{algocf}{section}
\numberwithin{equation}{section}
\numberwithin{lemmacounter}{section}
\numberwithin{definitioncounter}{section}

\begin{document}
    \newcommand{\courseshort}{\textbf{CS 437}}
\newcommand{\coursetitle}{Approximation Algorithms}
\newcommand{\term}{Fall Quarter 2025}
\newcommand{\prof}{Konstantin Makarychev}
\newcommand{\textbook}{???????}
\newcommand{\isbn}{??}


    \pagestyle{fancy}
    \fancyfoot[C]{}
    \fancyhead[L]{\textbf{Andrew Li} (\term)}
    \fancyfoot[L]{\courseshort \: \coursetitle}
    \fancyfoot[R]{\thepage}

    \title{\textbf{\textsf{\courseshort \, Lecture Notes}}}
    \author{Andrew Li}
    \date{\term}
    \clearpage\maketitle
    \noindent Original lecture notes for \textbf{\courseshort: \coursetitle}, from \term, taught by Professor \prof. This course follows \textbook, ISBN \isbn.
    \thispagestyle{empty}

    \tableofcontents
    \AddToHook{cmd/section/before}{\clearpage}
    
    \section{September 16, 2025}
I joined this class after this lecture. 

\subsection{Macros}
Below is an example algorithm using the macros in this repository. For simplicity, this algorithm computes the largest element of a fixed size array.

\begin{algorithm}[H]
    \caption{Algorithm to compute $\max(\text{list})$}
    \DontPrintSemicolon
    \textbf{input} list\;
    $curmax \leftarrow list[0]$\;
    \For{$n \in list$}{
        $curmax \leftarrow \max(n, curmax)$
    }
    \Return{curmax}
\end{algorithm}

There are also other environments, namely
\begin{lemma}
    This is a lemma.
\end{lemma}
\begin{proposition}
    and a proposition.
\end{proposition}
\begin{definition}
    and a definition.
\end{definition}
\begin{example}
    These boxes are for examples.
\end{example}
\begin{aside}
    These boxes are sparingly used, for asides.
\end{aside}
\begin{theorem}
    And finally, we've got the theorem.
\end{theorem}
As is standard, we can use the proof environment for proofs.
\begin{proof}
    Trivial.
\end{proof}

\subsection{Set Cover}
\begin{definition}
    \textbf{Set Cover} Let $V$ be some universe, with $\abs{V} = n$. Let
    \begin{align}
        S_1, \ldots, S_m \subseteq V
    \end{align}
    such that $\bigcup_i S_i = V$. Select the smallest $I \subseteq \{ 1, \ldots, m \}$ such that $\bigcup_{i \in I} S_i = V$.
\end{definition}

\begin{example}
    Let $V \equiv \{ 1, 2, 3, 4, 5 \}$ and sets be pairs $\{ i, j \}$ such that $i \ne j$. Then, an optimal solution is
    \begin{align}
        I \equiv \left\{ \{ 1, 2 \}, \{ 3, 4 \}, \{ 1, 5 \} \right\}
    \end{align}
    In this case, $\text{opt}(I) = 3$
\end{example}

\begin{definition}
    The approximation factor of an algorithm is $\alpha_n$ if for every $I$ of size $n$, we have
    \begin{align}
        \text{alg}(I) \le \alpha_n \cdot \text{opt}(I)
    \end{align}
\end{definition}

The first theorem of this course is
\begin{theorem}
    There exists a polynomial time algorithm with approximation factor $\log n$.
    \begin{algorithm}[H]
        \caption{Polynomial time set cover approximation algorithm}
        \DontPrintSemicolon
        $U_0 \leftarrow V$ \tcp{set of not yet covered elements in $V$}
        $t \leftarrow 0$ \tcp{iteration counter}
        \For{$U_t \ne \emptyset$}{
            \text{Select $S_i$ from sets that maximises $\abs{S_i \cap U_t}$}\;
            \text{Include $S_i$ in soln}\;
            \text{$U_t \leftarrow U_{t} \setminus S_i$}\;
            \text{$t \leftarrow t + 1$}\;
        }
        \Return{soln}
    \end{algorithm}
\end{theorem}

\subsubsection{Proof}


    \section{September 18, 2025}

\subsection{Finishing Previous Proof}

Recall some universe $V$, some family of sets $S_1, \ldots, S_m \subseteq V$, want to minimise size of family that spans entire $V$.

\begin{aside}
    All solutions are feasible, as the algorithm stops when $U_t = \emptyset$, i.e. when the selected sets span $V$. If there is no feasible solution, then the algorithm can just terminate when there are no more sets to select.
\end{aside}

Recall $k$ is the number in the optimal solution.

\begin{lemma}
    For $t^* = k \log n$,
    \e{
        \qty(1 - \frac{1}{k})^{t^*} < \frac{1}{n}
    }
    If this is true, then
    \e{
        \abs{U_{t^*}} < \frac{1}{n} \cdot n < 1
    }
    which implies $\abs{U_{t^*}} \equiv 0$. That imposes an upper bound on the time steps $t$ needed to cover all elements.
\end{lemma}

\begin{proof}
    Use the well-known definition of $e$
    \e{
        \qty(1 - \frac{1}{k})^{k\log n} &= \qty(\qty(1 - \frac{1}{k})^k)^{\log n} \\
        &< \qty(1/e)^{\log n}\\
        &< 1/n
    }
\end{proof}

\begin{aside}
    When $x \approx 0$,
    \e{
        e^{-x} \approx 1-x
    }
    In general,
    \e{
        1 - x < e^{-x}
    }
\end{aside}

\subsection{Weighted Set Cover Problem}

\begin{definition}
    \textbf{Weighted Set Cover Problem}
    Let $V$ be some universe, $S_1, \ldots, S_m \subseteq V$. Select sets of minimum cost that cover $V$, where set $S_i$ has cost/weight $w_i$.
\end{definition}

WLOG, we can assume strictly-positive costs (zero cost can be dealt with in pre-processing).

\begin{theorem}
    The algorithm for this is the same as before, but we select sets differently. We cannot ignore the cost.

    \begin{algorithm}[H]
        \caption{Polynomial time set cover approximation algorithm}
        \DontPrintSemicolon
        $U_0 \leftarrow V$ \tcp{set of not yet covered elements in $V$}
        $t \leftarrow 0$ \tcp{iteration counter}
        \For{$U_t \ne \emptyset$}{
            \text{Select $S_i$ from sets that maximises new elements per cost $\frac{\abs{S_i \cap U_t}}{w_i}$}\;
            \text{Include $S_i$ in soln}\;
            \text{$U_t \leftarrow U_{t} \setminus S_i$}\;
            \text{$t \leftarrow t + 1$}\;
        }
        \Return{soln}
    \end{algorithm}

    So we maximise new elements per cost, or minimise cost per new element.
\end{theorem}

\begin{proof}
    Prove by induction, on $\abs{U_t} \le \qty(1 - \frac{1}{k})^t \cdot n$. In this problem, that is analogous to
    \e{
        \abs{U_t} \le \exp\qty(-\frac{W_t}{\text{opt}}) \cdot n
    }
    Base case, if $t = 0$, then $w_t = 0$ and obviously
    \e{
        \abs{U_0} = n
    }
    Inductive step, assume inequality holds for some $t$,
    \e{
        \abs{U_t} \le \exp\qty(-\frac{W_t}{\text{opt}}) \cdot n
    }
    we can prove for $t+1$
    \e{
        \abs{U_{t+1}} \le \exp\qty(-\frac{W_{t+1}}{\text{opt}}) \cdot n
    }
    Let $S_{i_t}$ be the set we select at step $t$. Then
    \e{
        \frac{\abs{S_{i_t} \cap U_t}}{w_{i_t}}
    }
    is as large as possible per the greedy algorithm.
    \begin{lemma}
        \textbf{Claim}
        \e{
            \frac{\abs{S_{i_t} \cap U_t}}{w_{i_t}} \ge \frac{\abs{U_t}}{\text{opt}}
        }
    \end{lemma}
    \begin{proof}[Proof of Claim]
        Let $I^*$ be the set of indices of sets in $\texttt{opt}$.
        \e{
            \bigcup_{i \in I^*} S_i \cap U_t = U_t
        }
        This was proven earlier. Based on the proof from before,
        \e{
            \sum_{i \in I^*} \frac{\abs{S_i \cap U_t}}{\text{opt}}
            \ge \frac{\abs{U_t}}{\text{opt}}
        }
        This expands into
        \e{
            \sum_{i \in I^*} \frac{\abs{S_i \cap U_t}}{w_i} \cdot \frac{w_i}{\text{opt}}
            \ge \frac{\abs{U_t}}{\text{opt}} 
        }
        What if we only sum the $w_i / \text{opt}$? We get $1$. This above dot product is then a weighted sum of elements per cost. We conclude that
        \e{
            \exists i \qsp \frac{\abs{S_i \cap U_t}}{w_i} \ge \frac{U_t}{\text{opt}}
        }
        The greedy algorithm will choose the maximum so it will pick this $S_i$.
    \end{proof}
    
    Then,
    \e{
        \abs{U_{t+1}}
            &= \abs{U_t} - \abs{(S_{i_t} \cap U_t)}\\
            &\le n \cdot e^{-W_t / \text{opt}} \qty(1 - \frac{w_{i_t}}{\text{opt}})\\
            &\le n \cdot e^{-W_t / \text{opt}} \cdot e^{-w_{i_t} / \text{opt}} \\
            &\le n \cdot e^{-W_{t+1} / \text{opt}}
    }
    completing the proof.
\end{proof}

\subsection{Similar Problems}
Instead of covering all elements, try to cover as many elements as possible
\begin{definition}
    \textbf{Max $k$ Coverage}
    Choose $k$ sets to cover as many elements as possible. Can just look at the unweighted case.
\end{definition}

\subsection{Submodular Maximisation}

Take some set $X$, and some subsets $2^X$. Let
\e{
    f : 2^X \longrightarrow \real^+
}

\begin{example}
    Let $A \subseteq X$, $S_1, \ldots \in A$.
    Let $f$ be the coverage function,
    \e{
        f(A) = \abs{\bigcup_{S \in A} S}
    }
\end{example}

Take $A, B \subseteq X$. Obviously,
\e{
    f(A \cup B) \le f(A) + f(B)
}
is always true.

\begin{definition}
    \textbf{Subadditive Function}
    A function
    \e{
        f: 2^X \longrightarrow \real^+
    }
    is subadditive if
    \e{
        f(A) + f(B) \ge f(A \cup B)
    }
\end{definition}

\begin{definition}
    \textbf{Submodular Function}
    A function
    \e{
        f: 2^X \longrightarrow \real^+
    }
    is submodular if
    \e{
        f(A) + f(B) \ge f(A \cup B) + f(A \cap B)
    }
\end{definition}

All submodular functions are also subadditive.


    \section{September 23, 2025}

\subsection{Submodular Maximisation (cont)}
Recall the set cover problem, with some universe, and some set of sets that covers the entire universe. Now, maybe we want to pick a minimal subset that covers the entire universe. Or, we want to pick $k$ sets and maximise the coverage of the universe. Recall the definitions

\begin{definition}
    \textbf{Subadditive Function}
    A function
    \e{
        f: 2^X \longrightarrow \real^+
    }
    is subadditive if
    \e{
        f(A) + f(B) \ge f(A \cup B)
    }
\end{definition}

\begin{definition}
    \textbf{Submodular Function}
    A function
    \e{
        f: 2^X \longrightarrow \real^+
    }
    is submodular if
    \e{
        f(A) + f(B) \ge f(A \cup B) + f(A \cap B)
    }
\end{definition}

Equivalently, we can write
\begin{definition}
    \textbf{Submodular Function}
    A function
    \e{
        f : s^X \longrightarrow \real^+
    }
    is submodular if for two subsets $T \subseteq S \subset X$, and some $x \in X \setminus S$. The submodular property gives
    \e{
        f(T \cup \{ x \}) - f(T) \ge f(S \cup \{ x \}) - f(S)
    }
    A concrete example of this is diminishing utility of some commodity (e.g. money, some ETF, bananas).
\end{definition}

\begin{proposition}
    These definitions of submodular function are indeed equivalent.
\end{proposition}

\begin{proof}
    We can prove this as $\circled{3.2} \iff \circled{3.3}$
    \begin{enumerate}
        \item[($3.2 \implies 3.3$)] We want to prove
            \e{
                f(T \cup \{ x \}) - f(T) \stackrel{?}{\ge} f(S \cup \{ x \}) - f(S)
            }
            which is equivalent to
            \e{
                f(T \cup \{ x \}) + f(S) \stackrel{?}{\ge} f(S \cup \{ x \}) + f(T)
            }
            Let $A = T \cup \{ x \}$ and $B = S$. Then,
            \e{
                \qty(T \cup \{ x \}) \cup S = S \cup \{ x \} = A \cup B\\
                \qty(T \cup \{ x \}) \cap S = T = A \cap B
            }
            (This is trivial to see with a picture) Thus, the second definition is a consequence of the first.
        \item[($3.2 \impliedby 3.3$)] We want to prove
            \e{
                f(A \cup B) - f(A) \le f(B) - f(A \cap B)
            }
            which is pretty obviously equivalent to the first definition. To show this, initially, set $S = \emptyset$, then grow it to $S = B \setminus A$ one element at a time, which follows from the second definition. Then, the end result is
            \e{
                f(S \cup A) - f(A) \le f\qty(S \cup (A \cap B)) - f(A \cap B)
            }
            which for $S = B \setminus A$, is equivalent to
            \e{
                f(A \cup B) - f(A) \le f(B) - f(A \cap B)
            }
            (This is true via some very basic basic set theory, but is not basic to see. It is more clear with a picture)
    \end{enumerate}
\end{proof}

\subsection{Back to Coverage Functions}
Let $U$ be some universe, with subsets $S_1, \ldots, S_m \subseteq U$, and $X = \{ S_1, \ldots, S_m \}$. Let the coverage function
\e{
    f(A \in X) = \abs{\bigcup_{S_i \in A} S_i}
}
Concretely, suppose we have
\e{
    f(\{ S_1, S_3 \} \cup \{ S_2 \}) - f(\{ S_1, S_3 \}) = \abs{S_2 \setminus S_1 \setminus S_3}
}
If we also have $S_4$, then
\e{
    f(\{ S_1, S_3, S_4 \} \cup \{ S_2 \}) - f(\{ S_1, S_3, S_4 \}) = \abs{S_2 \setminus S_1 \setminus S_3 \setminus S_4} \le \abs{S_2 \setminus S_1 \setminus S_3}
}
i.e. there are `fewer elements' in the delta of the coverage function. (This is an obvious example, but it is still quite concrete and thus useful.)

\subsection{Maximisation}
\begin{theorem}
    The greedy algorithm gives a $1 - e^{-1}$ approximation for monotone submodular maximisation problem in which we need to select a given number of elements.
\end{theorem}
\begin{aside}
    Let $f$ be some monotone function such that
    \e{
        f(S \cup \{ x \}) \ge f(S)
    }
    Goal is to select $k$ elements $x_1, \ldots, x_k$ to maximise
    \e{
        f(\{ x_1, \ldots, x_k \})
    }
    Assume that $f(\emptyset) \equiv 0$
\end{aside}
\begin{proposition}
    We specifically want to prove
    \e{
        \texttt{ALG} \ge \qty(1 - \frac{1}{e}) \cdot \texttt{OPT}
    }
\end{proposition}
\begin{proof}
    Denote $\texttt{ALG}_i$ as the value that the greedy algorithm gets after $i$ steps. Also denote
    \e{
        \Lambda \equiv \{ x_1^*, \ldots, x_k^* \}
    }
    be an optimal solution. Finally, denote
    \e{
        \Delta_i \equiv \texttt{OPT} - \texttt{ALG}_i
    }
    which we can show shrinks fast. We thus need
    \e{
        \texttt{ALG}_{i+1} - \texttt{ALG}_i = ?
    }
    Suppose the algorithm has some set $A_j$ at step $i$. Then,
    \e{
        \texttt{ALG}_i \equiv f(A_j) \implies f(A_i \cup \Lambda) \ge \texttt{OPT}
    }
    Every time we add some $x_j^* \in \Lambda$, we are always lower-bounded on $f$ by $\texttt{OPT}$.
    \e{
        f(A_j \cup \{ x_1^*, \ldots, x_{i+1}^* \}) - f(A_j \cup \{ x_1^*, \ldots, x_i^* \}) \ge \frac{\texttt{OPT} - \texttt{ALG}_j}{k}
    }
    We can use some submodular math to rewrite
    \e{
        f(A_j \cup \{ x_{i+1}^* \}) - f(A_j) \ge \texttt{above}
    }
    We ultimately conclude that
    \e{
        f(A_{j+1}) - f(A_j) \ge \frac{\texttt{OPT} - \texttt{ALG}_j}{k}
    }
    The desired result thus `follows very easily'\footnote{does it?}.
    \e{
        \texttt{OPT} - f(A_{j+1}) &\le \texttt{OPT} - \qty(f(A_j) + \frac{\texttt{OPT} - \texttt{ALG}_j}{k}) \\
        &= \qty(\texttt{OPT} - f(A_j)) \cdot \qty(1 - 1/k)
    }
    After $k$ steps, the upper-bound becomes
    \e{
        \qty(\qty(1 - 1/k)^k \le 1/e) \cdot \texttt{OPT}
    }
    So we finally get
    \e{
        f(A_k) \ge \texttt{OPT} \cdot \qty(1 - 1/e)
    }
\end{proof}


    \section{September 25, 2025}

The professor started the class with a subtle comment on inflation since 1964.

\subsection{Travelling Salesman Problem (TSP)}

\begin{definition}
    Let $X$ be some universe, and $x, y \in X$. Let $d(x, y)$ be some distance function. Denote $(X, d)$ as a metric space if
    \begin{enumerate}
        \item $d(x, x) = 0$
        \item Positivity $d(x, y) > 0$
        \item Symmetry $d(x, y) = d(y, x)$
        \item Triangle Inequality $d(x, z) \le d(x, y) + d(y, z)$
    \end{enumerate}
\end{definition}

\begin{definition}
    \textbf{Metric TSP}

    Let $(X, d)$ be some metric space. Find the shortest tour that visits $x \in X$ exactly once.
\end{definition}

There exists a $3/2$-approximation for this problem. First, find a minimum spanning tree for the graph (which is trivial to do in polynomial time).
\begin{align}
    \text{cost}(\texttt{MST}) < \texttt{OPT}
\end{align}

How can we transform the tree into a tour? We can do a depth-first search (which gives us a pre-order traversal).
\begin{align}
    \texttt{OPT} \le \qty (
        \texttt{ALG} = 2 \cdot \text{cost}(\texttt{MST})
    ) < 2 \cdot \texttt{OPT}
\end{align}

This comes from visiting every edge twice. We can improve this algorithm by augmenting the tree as
\begin{align}
    \underbracket{T}_\text{MST} \cup \underbracket{M}_\text{set of edges}
\end{align}
such that we visit every edge exactly once, and then the new cost equals
\begin{align}
    \text{cost}(\texttt{ALG}) = \underbrace{\text{cost}(T)}_{\le \texttt{OPT}} + \underbrace{\text{cost}(M)}_{\stackrel{?}{\le} \texttt{OPT}/2}
\end{align}

This problem is similar to taking some graph and visiting each edge exactly once. That sounds quite a bit like finding an Eulerian cycle.

\subsection{Eulerian Graphs}

\begin{definition}
    A connected graph $G$ is Eulerian if there exists a \textit{tour} (start and end at the same node) that visits every edge exactly once.
\end{definition}

\begin{proposition}
    All vertices in an Eulerian graph must have an even degree.
\end{proposition}

\begin{theorem}
    If in a connected graph $G$, all vertices have an even degreee, then $G$ is Eulerian.
\end{theorem}

\begin{proof}
    Prove by induction on the number of edges. Let 3 edges be the base case, then a triangle is an Eulerian cycle. Inductively let $G$ be Eulerian and have $k$ edges, then for $k + 1$ edges, split the graph into some connected components, and by the inductive hypothesis, there exists a tour for each, so there exists a tour across all of them too.\footnote{The professor proved it this way... and by waving his hands.}
\end{proof}

\subsection{Christofides Algorithm}

\begin{enumerate}
    \item First, find an MST $T$
    \item Then, take some subset $S \subset G$ of all odd-degree vertices
    \item The number of elements in $S$ must be even, because if it was odd, then the sum of all degrees would be odd, and that is impossible for any graph because all edges are counted exactly twice.
    \item Find minimum cost matching for $S$, denoted as $M$
    \item Find Eulerian tour of $T \cup M$
\end{enumerate}

The proof of this was literally the professor waving his hands. It was not formalised.

    \section{September 30, 2025}

\subsection{Knapsack}
Very widespread problem.
\begin{definition}
    Given $N$ items with weights $w_1, \ldots, w_N$ and values $v_1, \ldots, v_N$, select a subset $I$ of weight at most $W$, where
    \begin{align}
        W(I) = \sum_{i \in I} w_i \le W
    \end{align}
    Maximise the value $V$, defined as
    \begin{align}
        V(I) = \sum_{i \in I} v_i
    \end{align}
    Assume $w_i > 0$, $v_i > 0$, as the zero cases are trivial: never take zero value, always take zero weight. Also assume $w_i \le W$.
\end{definition}

\begin{lemma}
    When all weights are equal $w_1 = \cdots = w_N$, greedily take the most valuable items. When all values are equal $v_1 = \cdots = v_N$, greedily take the least heavy items.
    \begin{proof}
        Trivial
    \end{proof}
\end{lemma}


\begin{aside}
    Haven't we seen this course at the undergraduate level? Yes and we can solve it with dynamic programming. Why do we need an approximation? Because the complexity of the dp algorithm is polynomial in $O(w \cdot N)$.
    \begin{enumerate}
        \item If $W \gg 0$ or $N \gg 0$, then the algorithm is too slow.
        \item If weights/values are floating point, then the algorithm won't be able to index a dp table.
    \end{enumerate}
\end{aside}

\subsubsection{Ideas}
\begin{enumerate}
    \item Pick the highest value density $\rho_i \equiv v_i/w_i$
    \item Naively pick the most valuable item (not that great, could be very heavy)
\end{enumerate}

If items can be taken fractionally, then the most optimal solution is to pick items with the highest value density, and take fractionally if that item's weight is too high.
\begin{definition}
    \textbf{Fractional Knapsack Problem} For some items with weights and values, a fraction $r \in [0, 1)$ of an item $i$ can be taken with weight $r\cdot w_i$ and value $r \cdot v_i$.
\end{definition}

\begin{proposition}
    The greedy algorithm based on highest value-density finds the optimal solution for the fractional knapsack problem.
\end{proposition}

\begin{proof}
    Let $\rho_1 > \rho_2 > \cdots > \rho_N$ be the sorted value-densities, such that this algorithm can be run in $O(N)$ time. Let
    \begin{align}
        a_1, a_2, \cdots, a_i, a_{i+1}, \cdots, a_N
    \end{align}
    such that $a_1 = a_2 = \cdots = a_i = 1$, and $a_{i+1} \in [0, 1)$, and $a_{i+2}, \cdots, a_N = 0$, be the fractions of the items we take. Let
    \begin{align}
        o_1, o_2, \cdots, o_N
    \end{align}
    be the optimal solution. Let $j$ be the first index where $a_j \ne o_j$, and $a_j > o_j$ (if less, then that doesn't make sense, because optimal implies space remains). Let
    \begin{align}
        \Delta \texttt{OPT} &= \Delta w_j \cdot \rho_j - \sum_{k = j+1}^N \Delta w_k \cdot \rho_k \\
        &> \Delta w_j \cdot \rho_j - \sum_{k = j+1}^N \Delta w_k \cdot \color{red} \rho_j \\
        &> \rho_j \qty(\Delta w_j - \sum_{k= j +1}^N \Delta w_k) \\
        &> \rho_j \Delta W \\
        &> 0
    \end{align}
    because the total weight should remain the same to fill the knapsack. Inductively because this implies that \texttt{OPT} would be non-zero improved, then the optimal solution must equal the algorithmic solution to avoid a contradiction.
\end{proof}

\begin{example}
    Let $v_1, w_1 = 1000, 1/10^6$ and $v_2, w_2 = 10^6, 1$, and $W = 1$. Then the highest-density algorithm fails, as it won't take the most valuable item that fits exactly.

\end{example}

For the non-fractional case, we can assume we take the entire fractional item. Formally,
\begin{align}
    \texttt{ALG}_1 + v_k &\ge \texttt{OPT}_\text{frac} \ge \texttt{OPT} \\
    \texttt{ALG}_2 &\ge v_k
\end{align}
where $\texttt{ALG}_1$ picks the highest-density item first while $\texttt{ALG}_2$ picks the single most valuable item first.

\begin{proposition}
    The best of $\texttt{ALG}_1$ and $\texttt{ALG}_2$ gives a 0.5 approximation.
\end{proposition}

\begin{proof}[Polynomial Time Approximation Scheme]

Let $\delta = \varepsilon / N$, and let
\begin{align}
    \texttt{OPT} \le M \le 2\texttt{OPT}
\end{align}
Round up every value $v_i$ to a multiple of $\delta M$ such that
\begin{align}
    v_i' = \left\lceil \frac{v_i}{\delta M} \right\rceil \cdot \delta M
\end{align}

\end{proof}

Next time we will finish this proof.

    % \include{contents/practice_final.tex}

\end{document}
