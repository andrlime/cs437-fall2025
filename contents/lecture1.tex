\section{September 16, 2025}
I joined this class after this lecture. 

\subsection{Macros}
Below is an example algorithm using the macros in this repository. For simplicity, this algorithm computes the largest element of a fixed size array.

\begin{algorithm}[H]
    \caption{Algorithm to compute $\max(\text{list})$}
    \DontPrintSemicolon
    \textbf{input} list\;
    $curmax \leftarrow list[0]$\;
    \For{$n \in list$}{
        $curmax \leftarrow \max(n, curmax)$
    }
    \Return{curmax}
\end{algorithm}

There are also other environments, namely
\begin{lemma}
    This is a lemma.
\end{lemma}
\begin{proposition}
    and a proposition.
\end{proposition}
\begin{definition}
    and a definition.
\end{definition}
\begin{example}
    These boxes are for examples.
\end{example}
\begin{aside}
    These boxes are sparingly used, for asides.
\end{aside}
\begin{theorem}
    And finally, we've got the theorem.
\end{theorem}
As is standard, we can use the proof environment for proofs.
\begin{proof}
    Trivial.
\end{proof}

\subsection{Set Cover}
\begin{definition}
    \textbf{Set Cover} Let $V$ be some universe, with $\abs{V} = n$. Let
    \begin{align}
        S_1, \ldots, S_m \subseteq V
    \end{align}
    such that $\bigcup_i S_i = V$. Select the smallest $I \subseteq \{ 1, \ldots, m \}$ such that $\bigcup_{i \in I} S_i = V$.
\end{definition}

\begin{example}
    Let $V \equiv \{ 1, 2, 3, 4, 5 \}$ and sets be pairs $\{ i, j \}$ such that $i \ne j$. Then, an optimal solution is
    \begin{align}
        I \equiv \left\{ \{ 1, 2 \}, \{ 3, 4 \}, \{ 1, 5 \} \right\}
    \end{align}
    In this case, $\text{opt}(I) = 3$
\end{example}

\begin{definition}
    The approximation factor of an algorithm is $\alpha_n$ if for every $I$ of size $n$, we have
    \begin{align}
        \text{alg}(I) \le \alpha_n \cdot \text{opt}(I)
    \end{align}
\end{definition}

The first theorem of this course is
\begin{theorem}
    There exists a polynomial time algorithm with approximation factor $\log n$.
    \begin{algorithm}[H]
        \caption{Polynomial time set cover approximation algorithm}
        \DontPrintSemicolon
        $U_0 \leftarrow V$ \tcp{set of not yet covered elements in $V$}
        $t \leftarrow 0$ \tcp{iteration counter}
        \For{$U_t \ne \emptyset$}{
            \text{Select $S_i$ from sets that maximises $\abs{S_i \cap U_t}$}\;
            \text{Include $S_i$ in soln}\;
            \text{$U_t \leftarrow U_{t} \setminus S_i$}\;
            \text{$t \leftarrow t + 1$}\;
        }
        \Return{soln}
    \end{algorithm}
\end{theorem}

\subsubsection{Proof}

