\section{September 16, 2025}
I joined this class after this lecture. 

\subsection{Macros}
Below is an example algorithm using the macros in this repository. For simplicity, this algorithm computes the largest element of a fixed size array.

\begin{algorithm}[H]
    \caption{Algorithm to compute $\max(\text{list})$}
    \DontPrintSemicolon
    \textbf{input} list\;
    $curmax \leftarrow list[0]$\;
    \For{$n \in list$}{
        $curmax \leftarrow \max(n, curmax)$
    }
    \Return{curmax}
\end{algorithm}

There are also other environments, namely
\begin{lemma}
    This is a lemma.
\end{lemma}
\begin{proposition}
    and a proposition.
\end{proposition}
\begin{definition}
    and a definition.
\end{definition}
\begin{example}
    These boxes are for examples.
\end{example}
\begin{aside}
    These boxes are sparingly used, for asides.
\end{aside}
\begin{theorem}
    And finally, we've got the theorem.
\end{theorem}
As is standard, we can use the proof environment for proofs.
\begin{proof}
    Trivial.
\end{proof}

\subsection{Set Cover}
\begin{definition}
    \textbf{Set Cover} Let $V$ be some universe, with $\abs{V} = n$. Let
    \e{
        S_1, \ldots, S_m \subseteq V
    }
    such that $\bigcup_i S_i = V$. Select the smallest $I \subseteq \{ 1, \ldots, m \}$ such that $\bigcup_{i \in I} S_i = V$.
\end{definition}

\begin{example}
    Let $V \equiv \{ 1, 2, 3, 4, 5 \}$ and sets be pairs $\{ i, j \}$ such that $i \ne j$. Then, an optimal solution is
    \e{
        I \equiv \left\{ \{ 1, 2 \}, \{ 3, 4 \}, \{ 1, 5 \} \right\}
    }
    In this case, $\text{opt}(I) = 3$
\end{example}

\begin{definition}
    The approximation factor of an algorithm is $\alpha_n$ if for every $I$ of size $n$, we have
    \e{
        \text{alg}(I) \le \alpha_n \cdot \text{opt}(I)
    }
\end{definition}

The first theorem of this course is
\begin{theorem}
    There exists a polynomial time algorithm with approximation factor $\log n$.

    \begin{algorithm}[H]
        \caption{Polynomial time set cover approximation algorithm}
        \DontPrintSemicolon
        $U_0 \leftarrow V$ \tcp{set of not yet covered elements in $V$}
        $t \leftarrow 0$ \tcp{iteration counter}
        \For{$U_t \ne \emptyset$}{
            \text{Select $S_i$ from sets that maximises $\abs{S_i \cap U_t}$}\;
            \text{Include $S_i$ in soln}\;
            \text{$U_t \leftarrow U_{t} \setminus S_i$}\;
            \text{$t \leftarrow t + 1$}\;
        }
        \Return{soln}
    \end{algorithm}
\end{theorem}

\subsubsection{Proof}
Let $k = \text{opt}$ be the number of sets in the optimal solution. Let $S_{i_1}$ be the first selected set. Then,
\e{
    \abs{S_{i_1}} \ge \frac{n}{k}
}
Then it follows that
\e{
    \abs{U_1} &= \abs{\underbracket{U_0}_{V} \setminus S_{i_1}} = \underbracket{\abs{U_0}}_{n} - \abs{S_{i_1}} \\
    &\le n - \frac{n}{k} = n\qty(1 - \frac{1}{k})
}
Let $S_{i_{t+1}}$ be the set chosen at iteration $t$.
\begin{lemma}
    \e{
        \bigcup_{i \in I^*} S_i \cap U_t = U_t
    }
\end{lemma}
\begin{proof}
    We can prove that LHS $\subseteq$ RHS and RHS $\subseteq$ LHS. To prove the first,
    \e{
        u \in \bigcup_{i \in I^*} S_i \cap U_t &\implies u \in \text{at least one $S_i \cap U_t$} \implies u \in U_t
    }
    Thus, every element in one of the chosen sets' intersection with $U_t$ is in $U_t$.
    \e{
        u \in U_t &\implies u \in \text{at least one $S_i$} & \text{$I^*$ spans universe; $S_i$ must exist}\\
        &\implies u \in \text{at least one $S_i \cap U_t$}\\
        &\implies u \in \bigcup_{i \in I^*} S_i \cap U_t
    }
    And, every element in $U_t$ is in at least one set.
\end{proof}

It follows that, because $S_i$ are not necessarily disjoint sets,
\e{
    \sum_{i \in I^*} \abs{S_i \cap U_t} \ge \abs{U_t}
}
Thus, given there are $k$ sets in $I^*$, by pigeonhole,
\e{
    \exists i \qsp \abs{S_i \cap U_t} \ge \frac{\abs{U_t}}{k}
}
Then,
\e{
    \abs{U_{t+1}}
        &= \abs{U_t \setminus (S_{i_{t+1}} \cap U_t)} \\
        &= \abs{U_t} - \abs{S_{i_{t+1}} \cap U_t} \\
        &\le \abs{U_t} - \frac{\abs{U_t}}{k} = \left( 1 - \frac{1}{k} \right) \abs{U_t}
}
Trivially,
\e{
    \abs{U_t} \le \left( 1 - \frac{1}{k} \right)^t \cdot n
}
\begin{proposition}
    For $t = k \log n$,
    \e{
        \left( 1 - \frac{1}{k} \right)^t < \frac{1}{n}
    }
\end{proposition}
