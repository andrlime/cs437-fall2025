\section{September 18, 2025}

\subsection{Finishing Previous Proof}

Recall some universe $V$, some family of sets $S_1, \ldots, S_m \subseteq V$, want to minimise size of family that spans entire $V$.

\begin{aside}
    All solutions are feasible, as the algorithm stops when $U_t = \emptyset$, i.e. when the selected sets span $V$. If there is no feasible solution, then the algorithm can just terminate when there are no more sets to select.
\end{aside}

Recall $k$ is the number in the optimal solution.

\begin{lemma}
    For $t^* = k \log n$,
    \e{
        \qty(1 - \frac{1}{k})^{t^*} < \frac{1}{n}
    }
    If this is true, then
    \e{
        \abs{U_{t^*}} < \frac{1}{n} \cdot n < 1
    }
    which implies $\abs{U_{t^*}} \equiv 0$. That imposes an upper bound on the time steps $t$ needed to cover all elements.
\end{lemma}

\begin{proof}
    Use the well-known definition of $e$
    \e{
        \qty(1 - \frac{1}{k})^{k\log n} &= \qty(\qty(1 - \frac{1}{k})^k)^{\log n} \\
        &< \qty(1/e)^{\log n}\\
        &< 1/n
    }
\end{proof}

\begin{aside}
    When $x \approx 0$,
    \e{
        e^{-x} \approx 1-x
    }
    In general,
    \e{
        1 - x < e^{-x}
    }
\end{aside}

\subsection{Weighted Set Cover Problem}

\begin{definition}
    \textbf{Weighted Set Cover Problem}
    Let $V$ be some universe, $S_1, \ldots, S_m \subseteq V$. Select sets of minimum cost that cover $V$, where set $S_i$ has cost/weight $w_i$.
\end{definition}

WLOG, we can assume strictly-positive costs (zero cost can be dealt with in pre-processing).

\begin{theorem}
    The algorithm for this is the same as before, but we select sets differently. We cannot ignore the cost.

    \begin{algorithm}[H]
        \caption{Polynomial time set cover approximation algorithm}
        \DontPrintSemicolon
        $U_0 \leftarrow V$ \tcp{set of not yet covered elements in $V$}
        $t \leftarrow 0$ \tcp{iteration counter}
        \For{$U_t \ne \emptyset$}{
            \text{Select $S_i$ from sets that maximises new elements per cost $\frac{\abs{S_i \cap U_t}}{w_i}$}\;
            \text{Include $S_i$ in soln}\;
            \text{$U_t \leftarrow U_{t} \setminus S_i$}\;
            \text{$t \leftarrow t + 1$}\;
        }
        \Return{soln}
    \end{algorithm}

    So we maximise new elements per cost, or minimise cost per new element.
\end{theorem}

\begin{proof}
    Prove by induction, on $\abs{U_t} \le \qty(1 - \frac{1}{k})^t \cdot n$. In this problem, that is analogous to
    \e{
        \abs{U_t} \le \exp\qty(-\frac{W_t}{\text{opt}}) \cdot n
    }
    Base case, if $t = 0$, then $w_t = 0$ and obviously
    \e{
        \abs{U_0} = n
    }
    Inductive step, assume inequality holds for some $t$,
    \e{
        \abs{U_t} \le \exp\qty(-\frac{W_t}{\text{opt}}) \cdot n
    }
    we can prove for $t+1$
    \e{
        \abs{U_{t+1}} \le \exp\qty(-\frac{W_{t+1}}{\text{opt}}) \cdot n
    }
    Let $S_{i_t}$ be the set we select at step $t$. Then
    \e{
        \frac{\abs{S_{i_t} \cap U_t}}{w_{i_t}}
    }
    is as large as possible per the greedy algorithm.
    \begin{lemma}
        \textbf{Claim}
        \e{
            \frac{\abs{S_{i_t} \cap U_t}}{w_{i_t}} \ge \frac{\abs{U_t}}{\text{opt}}
        }
    \end{lemma}
    \begin{proof}[Proof of Claim]
        Let $I^*$ be the set of indices of sets in $\texttt{opt}$.
        \e{
            \bigcup_{i \in I^*} S_i \cap U_t = U_t
        }
        This was proven earlier. Based on the proof from before,
        \e{
            \sum_{i \in I^*} \frac{\abs{S_i \cap U_t}}{\text{opt}}
            \ge \frac{\abs{U_t}}{\text{opt}}
        }
        This expands into
        \e{
            \sum_{i \in I^*} \frac{\abs{S_i \cap U_t}}{w_i} \cdot \frac{w_i}{\text{opt}}
            \ge \frac{\abs{U_t}}{\text{opt}} 
        }
        What if we only sum the $w_i / \text{opt}$? We get $1$. This above dot product is then a weighted sum of elements per cost. We conclude that
        \e{
            \exists i \qsp \frac{\abs{S_i \cap U_t}}{w_i} \ge \frac{U_t}{\text{opt}}
        }
        The greedy algorithm will choose the maximum so it will pick this $S_i$.
    \end{proof}
    
    Then,
    \e{
        \abs{U_{t+1}}
            &= \abs{U_t} - \abs{(S_{i_t} \cap U_t)}\\
            &\le n \cdot e^{-W_t / \text{opt}} \qty(1 - \frac{w_{i_t}}{\text{opt}})\\
            &\le n \cdot e^{-W_t / \text{opt}} \cdot e^{-w_{i_t} / \text{opt}} \\
            &\le n \cdot e^{-W_{t+1} / \text{opt}}
    }
    completing the proof.
\end{proof}

\subsection{Similar Problems}
Instead of covering all elements, try to cover as many elements as possible
\begin{definition}
    \textbf{Max $k$ Coverage}
    Choose $k$ sets to cover as many elements as possible. Can just look at the unweighted case.
\end{definition}

\subsection{Submodular Maximisation}

Take some set $X$, and some subsets $2^X$. Let
\e{
    f : 2^X \longrightarrow \real^+
}

\begin{example}
    Let $A \subseteq X$, $S_1, \ldots \in A$.
    Let $f$ be the coverage function,
    \e{
        f(A) = \abs{\bigcup_{S \in A} S}
    }
\end{example}

Take $A, B \subseteq X$. Obviously,
\e{
    f(A \cup B) \le f(A) + f(B)
}
is always true.

\begin{definition}
    \textbf{Subadditive Function}
    A function
    \e{
        f: 2^X \longrightarrow \real^+
    }
    is subadditive if
    \e{
        f(A) + f(B) \ge f(A \cup B)
    }
\end{definition}

\begin{definition}
    \textbf{Submodular Function}
    A function
    \e{
        f: 2^X \longrightarrow \real^+
    }
    is submodular if
    \e{
        f(A) + f(B) \ge f(A \cup B) + f(A \cap B)
    }
\end{definition}

All submodular functions are also subadditive.

