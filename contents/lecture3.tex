\section{September 23, 2025}

\subsection{Submodular Maximisation (cont)}
Recall the set cover problem, with some universe, and some set of sets that covers the entire universe. Now, maybe we want to pick a minimal subset that covers the entire universe. Or, we want to pick $k$ sets and maximise the coverage of the universe. Recall the definitions

\begin{definition}
    \textbf{Subadditive Function}
    A function
    \e{
        f: 2^X \longrightarrow \real^+
    }
    is subadditive if
    \e{
        f(A) + f(B) \ge f(A \cup B)
    }
\end{definition}

\begin{definition}
    \textbf{Submodular Function}
    A function
    \e{
        f: 2^X \longrightarrow \real^+
    }
    is submodular if
    \e{
        f(A) + f(B) \ge f(A \cup B) + f(A \cap B)
    }
\end{definition}

Equivalently, we can write
\begin{definition}
    \textbf{Submodular Function}
    A function
    \e{
        f : s^X \longrightarrow \real^+
    }
    is submodular if for two subsets $T \subseteq S \subset X$, and some $x \in X \setminus S$. The submodular property gives
    \e{
        f(T \cup \{ x \}) - f(T) \ge f(S \cup \{ x \}) - f(S)
    }
    A concrete example of this is diminishing utility of some commodity (e.g. money, some ETF, bananas).
\end{definition}

\begin{proposition}
    These definitions of submodular function are indeed equivalent.
\end{proposition}

\begin{proof}
    We can prove this as $\circled{3.2} \iff \circled{3.3}$
    \begin{enumerate}
        \item[($3.2 \implies 3.3$)] We want to prove
            \e{
                f(T \cup \{ x \}) - f(T) \stackrel{?}{\ge} f(S \cup \{ x \}) - f(S)
            }
            which is equivalent to
            \e{
                f(T \cup \{ x \}) + f(S) \stackrel{?}{\ge} f(S \cup \{ x \}) + f(T)
            }
            Let $A = T \cup \{ x \}$ and $B = S$. Then,
            \e{
                \qty(T \cup \{ x \}) \cup S = S \cup \{ x \} = A \cup B\\
                \qty(T \cup \{ x \}) \cap S = T = A \cap B
            }
            (This is trivial to see with a picture) Thus, the second definition is a consequence of the first.
        \item[($3.2 \impliedby 3.3$)] We want to prove
            \e{
                f(A \cup B) - f(A) \le f(B) - f(A \cap B)
            }
            which is pretty obviously equivalent to the first definition. To show this, initially, set $S = \emptyset$, then grow it to $S = B \setminus A$ one element at a time, which follows from the second definition. Then, the end result is
            \e{
                f(S \cup A) - f(A) \le f\qty(S \cup (A \cap B)) - f(A \cap B)
            }
            which for $S = B \setminus A$, is equivalent to
            \e{
                f(A \cup B) - f(A) \le f(B) - f(A \cap B)
            }
            (This is true via some very basic basic set theory, but is not basic to see. It is more clear with a picture)
    \end{enumerate}
\end{proof}

\subsection{Back to Coverage Functions}
Let $U$ be some universe, with subsets $S_1, \ldots, S_m \subseteq U$, and $X = \{ S_1, \ldots, S_m \}$. Let the coverage function
\e{
    f(A \in X) = \abs{\bigcup_{S_i \in A} S_i}
}
Concretely, suppose we have
\e{
    f(\{ S_1, S_3 \} \cup \{ S_2 \}) - f(\{ S_1, S_3 \}) = \abs{S_2 \setminus S_1 \setminus S_3}
}
If we also have $S_4$, then
\e{
    f(\{ S_1, S_3, S_4 \} \cup \{ S_2 \}) - f(\{ S_1, S_3, S_4 \}) = \abs{S_2 \setminus S_1 \setminus S_3 \setminus S_4} \le \abs{S_2 \setminus S_1 \setminus S_3}
}
i.e. there are `fewer elements' in the delta of the coverage function. (This is an obvious example, but it is still quite concrete and thus useful.)

\subsection{Maximisation}
\begin{theorem}
    The greedy algorithm gives a $1 - e^{-1}$ approximation for monotone submodular maximisation problem in which we need to select a given number of elements.
\end{theorem}
\begin{aside}
    Let $f$ be some monotone function such that
    \e{
        f(S \cup \{ x \}) \ge f(S)
    }
    Goal is to select $k$ elements $x_1, \ldots, x_k$ to maximise
    \e{
        f(\{ x_1, \ldots, x_k \})
    }
    Assume that $f(\emptyset) \equiv 0$
\end{aside}
\begin{proposition}
    We specifically want to prove
    \e{
        \texttt{ALG} \ge \qty(1 - \frac{1}{e}) \cdot \texttt{OPT}
    }
\end{proposition}
\begin{proof}
    Denote $\texttt{ALG}_i$ as the value that the greedy algorithm gets after $i$ steps. Also denote
    \e{
        \Lambda \equiv \{ x_1^*, \ldots, x_k^* \}
    }
    be an optimal solution. Finally, denote
    \e{
        \Delta_i \equiv \texttt{OPT} - \texttt{ALG}_i
    }
    which we can show shrinks fast. We thus need
    \e{
        \texttt{ALG}_{i+1} - \texttt{ALG}_i = ?
    }
    Suppose the algorithm has some set $A_j$ at step $i$. Then,
    \e{
        \texttt{ALG}_i \equiv f(A_j) \implies f(A_i \cup \Lambda) \ge \texttt{OPT}
    }
    Every time we add some $x_j^* \in \Lambda$, we are always lower-bounded on $f$ by $\texttt{OPT}$.
    \e{
        f(A_j \cup \{ x_1^*, \ldots, x_{i+1}^* \}) - f(A_j \cup \{ x_1^*, \ldots, x_i^* \}) \ge \frac{\texttt{OPT} - \texttt{ALG}_j}{k}
    }
    We can use some submodular math to rewrite
    \e{
        f(A_j \cup \{ x_{i+1}^* \}) - f(A_j) \ge \texttt{above}
    }
    We ultimately conclude that
    \e{
        f(A_{j+1}) - f(A_j) \ge \frac{\texttt{OPT} - \texttt{ALG}_j}{k}
    }
    The desired result thus `follows very easily'\footnote{does it?}.
    \e{
        \texttt{OPT} - f(A_{j+1}) &\le \texttt{OPT} - \qty(f(A_j) + \frac{\texttt{OPT} - \texttt{ALG}_j}{k}) \\
        &= \qty(\texttt{OPT} - f(A_j)) \cdot \qty(1 - 1/k)
    }
    After $k$ steps, the upper-bound becomes
    \e{
        \qty(\qty(1 - 1/k)^k \le 1/e) \cdot \texttt{OPT}
    }
    So we finally get
    \e{
        f(A_k) \ge \texttt{OPT} \cdot \qty(1 - 1/e)
    }
\end{proof}

